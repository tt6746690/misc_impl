% Template adapted from the Eurographics SGP 2016 template
\documentclass{6838publ}
\usepackage{6838}
% \input{\string~/.macros}

\SpecialIssuePaper
\electronicVersion 
\ifpdf \usepackage[pdftex]{graphicx} \pdfcompresslevel=9
\else \usepackage[dvips]{graphicx} \fi
\graphicspath{{./assets}}


 
\PrintedOrElectronic

\usepackage{t1enc,dfadobe}
\usepackage{egweblnk}
\usepackage{cite}
\usepackage{lipsum}
\usepackage{amsmath}
\usepackage{amsfonts}
\usepackage{amssymb}

\newcommand\sa{\ensuremath{\mathcal{a}}}
\newcommand\sd{\ensuremath{\mathcal{d}}}
\newcommand\se{\ensuremath{\mathcal{e}}}
\newcommand\sg{\ensuremath{\mathcal{g}}}
\newcommand\sh{\ensuremath{\mathcal{h}}}
\newcommand\si{\ensuremath{\mathcal{i}}}
\newcommand\sj{\ensuremath{\mathcal{j}}}
\newcommand\sk{\ensuremath{\mathcal{k}}}
\newcommand\sm{\ensuremath{\mathcal{m}}}
\newcommand\sn{\ensuremath{\mathcal{n}}}
\newcommand\so{\ensuremath{\mathcal{o}}}
\newcommand\sq{\ensuremath{\mathcal{q}}}
\newcommand\sr{\ensuremath{\mathcal{r}}}
\newcommand\st{\ensuremath{\mathcal{t}}}
\newcommand\su{\ensuremath{\mathcal{u}}}
\newcommand\sv{\ensuremath{\mathcal{v}}}
\newcommand\sw{\ensuremath{\mathcal{w}}}
\newcommand\sx{\ensuremath{\mathcal{x}}}
\newcommand\sy{\ensuremath{\mathcal{y}}}
\newcommand\sz{\ensuremath{\mathcal{z}}}
\newcommand\sA{\ensuremath{\mathcal{A}}}
\newcommand\sB{\ensuremath{\mathcal{B}}}
\newcommand\sC{\ensuremath{\mathcal{C}}}
\newcommand\sD{\ensuremath{\mathcal{D}}}
\newcommand\sE{\ensuremath{\mathcal{E}}}
\newcommand\sF{\ensuremath{\mathcal{F}}}
\newcommand\sG{\ensuremath{\mathcal{G}}}
\newcommand\sH{\ensuremath{\mathcal{H}}}
\newcommand\sI{\ensuremath{\mathcal{I}}}
\newcommand\sJ{\ensuremath{\mathcal{J}}}
\newcommand\sK{\ensuremath{\mathcal{K}}}
\newcommand\sL{\ensuremath{\mathcal{L}}}
\newcommand\sM{\ensuremath{\mathcal{M}}}
\newcommand\sN{\ensuremath{\mathcal{N}}}
\newcommand\sO{\ensuremath{\mathcal{O}}}
\newcommand\sP{\ensuremath{\mathcal{P}}}
\newcommand\sQ{\ensuremath{\mathcal{Q}}}
\newcommand\sR{\ensuremath{\mathcal{R}}}
\newcommand\sS{\ensuremath{\mathcal{S}}}
\newcommand\sT{\ensuremath{\mathcal{T}}}
\newcommand\sU{\ensuremath{\mathcal{U}}}
\newcommand\sV{\ensuremath{\mathcal{V}}}
\newcommand\sW{\ensuremath{\mathcal{W}}}
\newcommand\sX{\ensuremath{\mathcal{X}}}
\newcommand\sY{\ensuremath{\mathcal{Y}}}
\newcommand\sZ{\ensuremath{\mathcal{Z}}}

\newcommand\ba{\ensuremath{\mathbf{a}}}
\newcommand\bb{\ensuremath{\mathbf{b}}}
\newcommand\bc{\ensuremath{\mathbf{c}}}
\newcommand\bd{\ensuremath{\mathbf{d}}}
\newcommand\be{\ensuremath{\mathbf{e}}}
\newcommand\bg{\ensuremath{\mathbf{g}}}
\newcommand\bh{\ensuremath{\mathbf{h}}}
\newcommand\bi{\ensuremath{\mathbf{i}}}
\newcommand\bj{\ensuremath{\mathbf{j}}}
\newcommand\bk{\ensuremath{\mathbf{k}}}
\newcommand\bl{\ensuremath{\mathbf{l}}}
\newcommand\bn{\ensuremath{\mathbf{n}}}
\newcommand\bo{\ensuremath{\mathbf{o}}}
\newcommand\bp{\ensuremath{\mathbf{p}}}
\newcommand\bq{\ensuremath{\mathbf{q}}}
\newcommand\br{\ensuremath{\mathbf{r}}}
\newcommand\bs{\ensuremath{\mathbf{s}}}
\newcommand\bt{\ensuremath{\mathbf{t}}}
\newcommand\bu{\ensuremath{\mathbf{u}}}
\newcommand\bv{\ensuremath{\mathbf{v}}}
\newcommand\bw{\ensuremath{\mathbf{w}}}
\newcommand\bx{\ensuremath{\mathbf{x}}}
\newcommand\by{\ensuremath{\mathbf{y}}}
\newcommand\bz{\ensuremath{\mathbf{z}}}
\newcommand\bA{\ensuremath{\mathbf{A}}}
\newcommand\bB{\ensuremath{\mathbf{B}}}
\newcommand\bC{\ensuremath{\mathbf{C}}}
\newcommand\bD{\ensuremath{\mathbf{D}}}
\newcommand\bE{\ensuremath{\mathbf{E}}}
\newcommand\bF{\ensuremath{\mathbf{F}}}
\newcommand\bG{\ensuremath{\mathbf{G}}}
\newcommand\bH{\ensuremath{\mathbf{H}}}
\newcommand\bI{\ensuremath{\mathbf{I}}}
\newcommand\bJ{\ensuremath{\mathbf{J}}}
\newcommand\bK{\ensuremath{\mathbf{K}}}
\newcommand\bL{\ensuremath{\mathbf{L}}}
\newcommand\bM{\ensuremath{\mathbf{M}}}
\newcommand\bN{\ensuremath{\mathbf{N}}}
\newcommand\bO{\ensuremath{\mathbf{O}}}
\newcommand\bP{\ensuremath{\mathbf{P}}}
\newcommand\bQ{\ensuremath{\mathbf{Q}}}
\newcommand\bR{\ensuremath{\mathbf{R}}}
\newcommand\bS{\ensuremath{\mathbf{S}}}
\newcommand\bT{\ensuremath{\mathbf{T}}}
\newcommand\bU{\ensuremath{\mathbf{U}}}
\newcommand\bV{\ensuremath{\mathbf{V}}}
\newcommand\bW{\ensuremath{\mathbf{W}}}
\newcommand\bX{\ensuremath{\mathbf{X}}}
\newcommand\bY{\ensuremath{\mathbf{Y}}}
\newcommand\bZ{\ensuremath{\mathbf{Z}}}
\newcommand\Ba{\ensuremath{\mathbb{a}}}
\newcommand\Bb{\ensuremath{\mathbb{b}}}
\newcommand\Bc{\ensuremath{\mathbb{c}}}
\newcommand\Bd{\ensuremath{\mathbb{d}}}
\newcommand\Be{\ensuremath{\mathbb{e}}}
\newcommand\Bf{\ensuremath{\mathbb{f}}}
\newcommand\Bg{\ensuremath{\mathbb{g}}}
\newcommand\Bh{\ensuremath{\mathbb{h}}}
\newcommand\Bi{\ensuremath{\mathbb{i}}}
\newcommand\Bj{\ensuremath{\mathbb{j}}}
\newcommand\Bk{\ensuremath{\mathbb{k}}}
\newcommand\Bl{\ensuremath{\mathbb{l}}}
\newcommand\Bm{\ensuremath{\mathbb{m}}}
\newcommand\Bn{\ensuremath{\mathbb{n}}}
\newcommand\Bo{\ensuremath{\mathbb{o}}}
\newcommand\Bp{\ensuremath{\mathbb{p}}}
\newcommand\Bq{\ensuremath{\mathbb{q}}}
\newcommand\Br{\ensuremath{\mathbb{r}}}
\newcommand\Bs{\ensuremath{\mathbb{s}}}
\newcommand\Bt{\ensuremath{\mathbb{t}}}
\newcommand\Bu{\ensuremath{\mathbb{u}}}
\newcommand\Bv{\ensuremath{\mathbb{v}}}
\newcommand\Bw{\ensuremath{\mathbb{w}}}
\newcommand\Bx{\ensuremath{\mathbb{x}}}
\newcommand\By{\ensuremath{\mathbb{y}}}
\newcommand\Bz{\ensuremath{\mathbb{z}}}
\newcommand\BA{\ensuremath{\mathbb{A}}}
\newcommand\BB{\ensuremath{\mathbb{B}}}
\newcommand\BC{\ensuremath{\mathbb{C}}}
\newcommand\BD{\ensuremath{\mathbb{D}}}
\newcommand\BE{\ensuremath{\mathbb{E}}}
\newcommand\BF{\ensuremath{\mathbb{F}}}
\newcommand\BG{\ensuremath{\mathbb{G}}}
\newcommand\BH{\ensuremath{\mathbb{H}}}
\newcommand\BI{\ensuremath{\mathbb{I}}}
\newcommand\BJ{\ensuremath{\mathbb{J}}}
\newcommand\BK{\ensuremath{\mathbb{K}}}
\newcommand\BL{\ensuremath{\mathbb{L}}}
\newcommand\BM{\ensuremath{\mathbb{M}}}
\newcommand\BN{\ensuremath{\mathbb{N}}}
\newcommand\BO{\ensuremath{\mathbb{O}}}
\newcommand\BP{\ensuremath{\mathbb{P}}}
\newcommand\BQ{\ensuremath{\mathbb{Q}}}
\newcommand\BR{\ensuremath{\mathbb{R}}}
\newcommand\BS{\ensuremath{\mathbb{S}}}
\newcommand\BT{\ensuremath{\mathbb{T}}}
\newcommand\BU{\ensuremath{\mathbb{U}}}
\newcommand\BV{\ensuremath{\mathbb{V}}}
\newcommand\BW{\ensuremath{\mathbb{W}}}
\newcommand\BX{\ensuremath{\mathbb{X}}}
\newcommand\BY{\ensuremath{\mathbb{Y}}}
\newcommand\BZ{\ensuremath{\mathbb{Z}}}
\newcommand\balpha{\ensuremath{\mbox{\boldmath$\alpha$}}}
\newcommand\bbeta{\ensuremath{\mbox{\boldmath$\beta$}}}
\newcommand\btheta{\ensuremath{\mbox{\boldmath$\theta$}}}
\newcommand\bphi{\ensuremath{\mbox{\boldmath$\phi$}}}
\newcommand\bpi{\ensuremath{\mbox{\boldmath$\pi$}}}
\newcommand\bpsi{\ensuremath{\mbox{\boldmath$\psi$}}}
\newcommand\bmu{\ensuremath{\mbox{\boldmath$\mu$}}}

\newcommand\R{\ensuremath{\mathbb{R}}} % Real numbers
\newcommand\Z{\ensuremath{\mathbb{Z}}} % Integers


\newcommand{\norm}[1]{\left\lVert#1\right\rVert}
\newcommand\inner[2]{\ensuremath{\left< #1, #2 \right>}} % Inner product
\DeclareMathOperator*{\diag}{diag} % Diagonal matrix
\newcommand\p[1]{\ensuremath{\left( #1 \right)}} % Parenthesis ()
\newcommand\pa[1]{\ensuremath{\left\langle #1 \right\rangle}} % <>
\newcommand\pb[1]{\ensuremath{\left[ #1 \right]}} % []
\newcommand\pc[1]{\ensuremath{\left\{ #1 \right\}}} % {}


\title[]{Optimal Transport based Probabilistic Diffeomorphic Registration}

\author[P.W.]
       {Peiqi Wang
        \\
        MIT Department of Electrical Engineering and Computer Science\\
       }

\begin{document}

\teaser{
 \includegraphics[width=.9\linewidth]{{assets/amoeba/plt_lddmm_points}}
 \centering
  \caption{The red amoeba is registered to blue amoeba. A valid diffeomorphic transformation can be generated from per-particle momenta (orange arrow) by solving a set of Geodesic equations (black dashed line). On the right, we can visualize variance of the effect of random transformation on shape as a heatmap. Although the two shapes is matched almost perfectly, the two legs flips over itself respectively and we are able to spot this mistake from the uncertainty heatmap.}
\label{fig:teaser}
}

\maketitle

\begin{abstract}

\end{abstract}

\section{Introduction}

 
% narrow in no topic: remind this is a graphics paper, need to help them to figure out what topic and area of research. no need to wax poetic about topic's importance


% dig a hole: convince reader there is a problem with the state of the world. prior work may exist but its missing something important or there is a missing opportunity. the reading should be drooling for a bright future just out of reach

Diffeomorphic registration of shapes with unknown correspondence is an important step in medical data processing. The choice of similarity metric for good matching distinguishes the different algorithms in this domain. Recent work explored the use of entropic regularized wasserstein distance as a global measure of similarity between encoding of shapes as discrete measures \cite{feydyOptimalTransportDiffeomorphic2017a,feydyFastScalableOptimal2019}. However, a point estimate of the transformation yield errors that may invalidate downstream processing pipelines or misguide clinical decision making. Additionally, solution is sensitive to hyperparameters of the problem, requiring manual tuning for each new shape. 

% fill the hole: to show reader how/why the paper will fix these problems and deliver us into a better place. don't need a whirlwind summary of technical details, but need reader's convinced to keep reading. 

We propose to extend optimal transport based diffeomorphic registration to probabilistic setting. Our method interprets the diffeomorphic transformation as a random variable, and estimates its parameters using variational inference. In particular, we find diffeomorphic maps that minimize the average Wasserstein distance between shapes. Naturally, the probabilistic formulation provides us with uncertainty estimates of both the transformation as well as the uncertainty of its effect on shapes. Hyperparameters such as the degree of smoothness of the transformation parameterizes the variational distribution, and thus can be optimized. However, the inference procedure requires repeated sampling of valid diffeomorphic transformations to approximate the average cost. To alleviate the computational burden, we explored links to sparse Gaussian Process, specifically interdomain inducing variables, as a way to alleviate this concern \cite{figueiras-vidalInterdomainGaussianProcesses2009a}.


\section{Related Work}

\subsection{Diffeomorphic Registration}

The large deformation registration of landmarks (point sets) and images as a variational problem that solves for a smooth time-varying velocity field that matches the two objects according to some measure of similarity \cite{joshiLandmarkMatchingLarge2000,begComputingLargeDeformation2005}. \cite{millerGeodesicShootingComputational2006,vialardDiffeomorphic3DImage2012} makes connection to optimal control, and showed that large deformation diffeormorphisms obeys conservation of momentum, and that points/images evolve according to a set of Geodesic Equations completely determined by its initial momentum. This observation prompted the development of geodesic shooting methods that optimizes for initial momentum for point sets and meshes \cite{vaillantStatisticsDiffeomorphismsTangent2004,allassonniereGeodesicShootingDiffeomorphic2005} and later extended to images \cite{vialardDiffeomorphic3DImage2012}. In our formulation, we represent a random diffeomorphic map as a pushforward of initial momentum, parameterized by some easy-to-work-with distribution.

\subsection{Optimal Transport}

The optimal transport problem give rise to a notion of distance between probability distributions. Intuitively, it measures the amount of mass needed to be displaced from one measure to another. Computation of such distance is useful in numerous graphics \cite{degoesOptimalTransportApproach2011,degoesBlueNoiseOptimal2012,solomonConvolutionalWassersteinDistances2015} and learning \cite{frognerLearningWassersteinLoss2015, janatiSpatioTemporalAlignmentsOptimal2020} applications. Developments in numerical methods in computing and differentiating through the optimal transport distance make for easier integration to a larger system \cite{cuturiSinkhornDistancesLightspeed2013,altschulerNearlinearTimeApproximation2018,genevayLearningGenerativeModels2018,schmitzerStabilizedSparseScaling2019}. A series of work \cite{feydyOptimalTransportDiffeomorphic2017a,feydyFastScalableOptimal2019} utilize unbalanced optimal transport distance to register shapes with possibly different mass. The motivation is to combine the elastic properties of diffeomorphic registration beneficial for medical data with a cost that is sensitive to both local and global shape features. We extend their methods to probabilistic settings.


\subsection{Probabilistic Registration}

Many probabilistic registration methods on image data relie on latent variable framework, where an unknown transformation is applied to the source image to generate the target image, corrupted by iid Gaussian noise. Previous, Monte Carlo methods have been used to estimate transformation parameters \cite{risholmBayesianCharacterizationUncertainty2013,zhangBayesianEstimationRegularization2013}. Although asymptotically exact, sampling methods can be slow for estimating the high dimensional parameters of a diffeomorphism. A few follow ups proposed work-arounds by reducing the number of parameters using factor analysis \cite{risholmBayesianCharacterizationUncertainty2013} or with an economical representation in the Fourier domain \cite{wangRegistrationUncertaintyQuantification2019}. Alternatively, variational inference has been proposed as a more tractable alternative for inference \cite{wassermannProbabilisticDiffeomorphicRegistration2014,dalcaUnsupervisedLearningProbabilistic2019}. Although similar in formulation, we differ from \cite{wassermannProbabilisticDiffeomorphicRegistration2014} as we restrict ourselves to diffeomorphic registration of discrete representation of shapes with unknown correspondence. Theory from geodesic shooting implies that we do not need to model the Eulerian flow as the solution to some stochastic differential equation as proposed in \cite{wassermannProbabilisticDiffeomorphicRegistration2014}. It suffices to model the initial momentum as some Gaussian Process whereby to specify a random diffeomorphic transformations.

% Descriptions of and citations to academic research papers and/or existing software products related to your work.  Here is an example of how to cite a paper~\cite{solomon-2016}; see \texttt{6838bibsample.bib} for bibliography entries.


\section{Preliminaries}\label{sec:preliminaries}


\subsection{Shape as Measures}

Let $\Omega \subset\R^D$ be a low dimensional ambient space. We consider a representation of shapes as discrete measures $\alpha = \sum_{i=1}^N a_i \delta_{x_i}, \beta = \sum_{j=1}^M b_i \delta_{y_j}$ where $a_i,b_j\in\R_+$ encodes some local statistics of shape, e.g. per-vertex length for curves or per-vertex area for surfaces, and $x := (x^1,\cdots,x^n)\subset \R^{N\times D}$, $y := (y^1,\cdots,y^n)\subset\R^{M\times D}$ are source and target points respectively. 


\subsection{Diffeomorphic Registration}
% We follow the usual setup for registering unlabeled point sets using geodesic shooting \cite{vaillantStatisticsDiffeomorphismsTangent2004}. 


The goal of diffeomorphic registration of point sets is to transform $x$ via a diffeomorphic mapping $\varphi$ such that the pushforward $\varphi_\sharp \alpha$ is close to $y$ according to some similarity metric $\sL(\varphi_\sharp \alpha, \beta)$. We consider a space of velocity fields $V$ as a RKHS over $\Omega$ characterized by kernel $\overline{k}:\Omega\times\Omega\to\R^{D\times D}$. For purposes of computation, we consider an equivalent scalar-valued kernel $k:\Omega\times[D]\times\Omega\times[D] \to\R$ where $\overline{k}(x,x')_{dd'} = k((x,d),(x',d'))$, inducing an RKHS that is isomorphic to $V$. A diffeomorphism can be constructed via flows , i.e. solutions to an ODE problem $\varphi_t$, $\dot{\varphi}_t = v_t\circ \varphi_t, \varphi_0 = \text{Id}$, of a sufficiently smooth velocity field, i.e. $\int_0^1 \norm{v_t}_V\, dt<\infty$. The large deformation registration problem solves for a time-varying velocity field $v_t\in V$ matching the two shapes,
\begin{align}
    \min_{v_t:t\in [0,1]} \,
        &\frac{1}{2} \int_0^1 \norm{v_t}_V^2 \, dt + \sL(\varphi_\sharp \alpha, \beta)
\end{align}

\subsection{Geodesic Shooting}

Let $q_t^i = \varphi_t(x^i) \in \R^D$ be application of transformation $\varphi_t$ to point $x^i$. Denote $q_t = (q_t^1, \cdots, q_t^N) \in \R^{ND}$ as action of $\varphi_t$ to a set of points and $K(q_t,q_t) \in \R^{ND\times ND}$ be kernel matrix for vectorized velocity vector field at $q_t$. \cite{joshiLandmarkMatchingLarge2000} argues for a Lagrangian view of previous Eulerian problem - it suffices to solve for the flow velocity $\dot{q}_t$ of particles,
\begin{align}
    \min_{\dot{q}_t:t\in [0,1]}
        &\frac{1}{2} \int_0^1 \inner{\dot{q}_t}{K(q_t,q_t)^{-1}\dot{q}_t} \, dt + \sL(\varphi_\sharp \alpha, \beta)
    \label{eq:optimization_lddmm_landmark_lagrangian}
\end{align}
and that the resulting flow velocity in the Eulerian coordinates $\Omega$ can be interpolated from $\dot{q}_t$,
\begin{align}
    v_t(x)
        = K(x, q_t) p_t
    \quad\quad
    p_t
        = K(q_t, q_t)^{-1} \dot{q}_t
\end{align}
where $p_t^i \in \R^{D}$ is the momenta associated with point $q_t^i$. As a side note, this interpolation is akin to computing posterior mean of a gaussian process regression of velocity fields, i.e. $v_t(x) = K(x,q_t) K(q_t,q_t)^{-1} \dot{q}_t$. Note the integrand of RKHS norm can be viewed as Lagrangian of of a system of ND particles. \cite{millerGeodesicShootingComputational2006} argues the dynamics of these particles in canonical coordinates $(q_t,p_t)$ follows the Geodesic equations
\begin{align}
    \dot{q}_t
        = \frac{\partial \sH(q_t,p_t)}{\partial p}
    \quad\quad
    \dot{p}_t
        = - \frac{\partial \sH(q_t,p_t)}{\partial q}
    \label{eq:geodesic_equations}
\end{align}
with initial condition $q_0 := x, p_0 := m$, and that the Hamiltonian 
\begin{align}
    \sH(q_t,p_t) = \frac{1}{2}\inner{p_t}{K(q_t,q_t)p_t} = \frac{1}{2}\inner{K(q_t,q_t)^{-1}\dot{q}_t}{\dot{q}_t}  
\end{align}
is preseved by the flow, $\sH(q_0,p_0)  = \sH(q_t,p_t)$ for all $t\in [0,1]$ (see Figure~(\ref{fig:plt_shooting})). Therefore, $\int_0^1 \sH(q_t,p_t) \, dt = \sH(q_0, p_0)$. Equvialent to (\ref{eq:optimization_lddmm_landmark_lagrangian}), we optimize over the initial \textit{shooting momentum} $m$,
\begin{align}
    \min_{m\in\R^{ND}} \,
        \frac{1}{2} \inner{ m }{ K(x,x) m } + \sL(\varphi_\sharp \alpha, \beta)
    \label{eq:optimization_lddmm_landmark_momentum}
\end{align}
where $\varphi_\sharp\alpha = \sum_{i=1}^N a_i' q_1^i$ and $a_i'$ is the updated weights as a function of updated vertex positions and unchanged topology. Optimization involves a forward integration of $(q_t,p_t)$ via (\ref{eq:geodesic_equations}) to get transformed particles $q_1$, compute the gradient of objective (\ref{eq:optimization_lddmm_landmark_momentum}) with respect to initial momentum, and do gradient update iteratively. 


\begin{center} 
\begin{figure}[t]
    \includegraphics[width=\linewidth]{assets/plt_shooting} 
    \caption{Geodesic shooting with an Euler integrator with time step of $\delta t = .1$. The velocity field is represented using a radial basis kernel with $\sigma=.25$. We show trajectory of $q_t$ (red dots) with momentum $p_t$ (blue arrow) and interpolated velocity fields at grid points (black). We see Hamiltonian is approximately conserved!}
    \label{fig:plt_shooting}
\end{figure} 
\end{center}

% $\alpha,\beta$ is not restricted to probability measures 

\subsection{Unbalanced Regularized Optimal Transport}

Define $c:\Omega^2 \to \R$ be cost of mass transportation. The entropic regularized unbalanced optimal transport problem seeks a soft coupling between measures $\mu,\nu$ supported over $\Omega$ that minimizes expected cost; The optimal value of which defines a measure of similarity between $\mu$ and $\nu$,
\begin{align}
    W_{\epsilon,\rho}(\mu,\nu)
        := \min_{\pi\in\sP(\Omega^2)}\,
            \iint_{\Omega^2} c(x,y)\, d\pi(x,y) + \epsilon \text{KL}(\pi \Vert \mu\otimes\nu) \\
                \quad\quad+\rho\left( \text{KL}(P^1_\sharp \pi \Vert \mu) + \text{KL}(P^2_\sharp \pi \Vert \nu) \right)
    \label{eq:ot_general_measures}
\end{align}
$P^i_\sharp$ is the projection to the $i$-th marginal. The correspondence of the optimal transport plan $\pi$ can be made sharper by using a smaller $\epsilon >0$. If the two shapes vary in scale and size, we can relax the mass conservation constraints by using a smaller $\rho>0$. For discrete measures, generalized Sinkhorn's algorithm provides efficient and stable numerical methods to compute $W_{\epsilon,\rho}(\alpha,\beta)$ \cite{chizatScalingAlgorithmsUnbalanced2017,feydyInterpolatingOptimalTransport2018}. The idea is to do coordinate ascent over the dual variable $u\in\R^N,v\in\R^M$,
\begin{align}
    u^{(\ell+1)}
        &\leftarrow \lambda\epsilon \left( \log(a) - \log(Ke^{v^{(\ell)}/\epsilon}) \right) \\
    v^{(\ell+1)}
        &\leftarrow \lambda\epsilon \left( \log(b) - \log(Ke^{u^{(\ell+1)}/\epsilon}) \right)
    \label{eq:sinkhorn_dual_ascent}
\end{align}
where $\lambda = \frac{\rho}{\rho+\epsilon}$. This algorithm, unlike the original matrix scaling algorithm, is numerically stable due to computation in log domain. A remarkable property is that gradients of $W_{\epsilon,\rho}(\alpha,\beta)$ with respect to $x_i,a_i$ is readily computable, prompting easy integration into existing gradient based methods for registration problems \cite{feydyOptimalTransportDiffeomorphic2017a}. 


\section{Approach}\label{sec:approach}

\subsection{Probabilistic Diffeomorphic Registration}










 



\section{Results}

Figures/tables illustrating the results of your work, as well as text interpreting these results. \cite{peyreComputationalOptimalTransport2020},


\bibliographystyle{eg-alpha-doi}
\bibliography{optimal_transport,registration,GP}


\end{document}
