\documentclass[11pt]{article}
\input{\string~/.macros}
\usepackage[a4paper, total={6in, 9in}]{geometry}
\usepackage{bbm}
\usepackage{mathrsfs} % really cursive alphabets
\usepackage{graphicx}
\graphicspath{{./assets}}
\usepackage{hyperref}
\hypersetup{colorlinks=true, linktoc=all, linkcolor=blue, citecolor=red}
\usepackage[backend=bibtex,sorting=none]{biblatex}


% random variables
\newcommand\ry{\ensuremath{\mathsf{y}}}
\newcommand\rx{\ensuremath{\mathsf{x}}}
\newcommand\rb{\ensuremath{\mathsf{b}}}
\newcommand\rc{\ensuremath{\mathsf{c}}}
\newcommand\rz{\ensuremath{\mathsf{z}}}
\newcommand\ru{\ensuremath{\mathsf{u}}}
\newcommand\rbx{\ensuremath{\mathsf{\mathbf{x}}}}
\newcommand\rby{\ensuremath{\mathsf{\mathbf{y}}}}
\newcommand\rbu{\ensuremath{\mathsf{\mathbf{u}}}}

% boldsymbols
\renewcommand\bmu{\ensuremath{\boldsymbol{\mu}}}
\newcommand\bSigma{\ensuremath{\boldsymbol{\Sigma}}}

% optimization, classes of functions
\newcommand\scrF{\ensuremath{\mathscr{F}}}
\newcommand\scrS{\ensuremath{\mathscr{S}}}

% independence
\newcommand{\dperp}{\ensuremath{\perp\!\!\!\perp}}
\newcommand{\ndperp}{\ensuremath{\not\!\perp\!\!\!\perp}}
\addbibresource{registration}
\addbibresource{optimal_transport}

\title{Project Proposal}
\author{Peiqi Wang (921301558)}

\begin{document}

\maketitle


I am planning to implement diffeomorphic registration with optimal transport cost \cite{feydyOptimalTransportDiffeomorphic2017a} and possibly extend it to probabilistic settings, referencing \cite{wassermannProbabilisticDiffeomorphicRegistration2014}.

Large deformation metric mapping is a registration algorithm that handles large deformation of objects and ensures the transformations are smooth and invertible \cite{begComputingLargeDeformation2005}. It solves for a time-varying velocity vector field $v_t: \Omega \to \R^n$ for $t\in [0,1]$ dictating the dynamic of a time-varying transformation $\phi_t:\Omega\to\Omega$, 
\begin{align}
    \frac{d}{dt} \phi_t = v_t(\phi_t)
    \quad\quad
    \phi_0 = \text{Id}
\end{align}
where $\Omega$ is ambient space, e.g. $\R^2$ when registering 2d images. The desired transformation is the end point of the above ODE problem, $\varphi = \phi_1 = \phi_0 + \int_0^1 v_t(\phi_t) \, dt$. It has been shown that if the velocities are sufficiently smooth, then $\varphi$ is a diffeomorphic map. Assume that $\mu = \sum_{i\in I} p_i \delta_{x_i}, \nu = \sum_{j\in J}q_j \delta_j$ is some discrete representation of shape $X,Y$. We are interested in finding a diffeomorphic transformation $\varphi$ s.t. the pushforward $\varphi_{\#}\mu$ is similar to the target $\nu$, captured by the following energy
\begin{align}
    \sE(\varphi)
        = \sR(\varphi) + \sL(\varphi_{\#}\mu, \nu)
\end{align}
where $\sR$ regularizes $\varphi$ be smooth and that $\sL$ is a data fidelity term. \cite{feydyOptimalTransportDiffeomorphic2017a} proposes to use a regularized unbalanced optimal transport cost between two discrete measures, i.e. $\sL := W_{\epsilon,\rho}$ where
\begin{align}
    W_{\epsilon,\rho}(\mu, \nu)
        = \min_{\gamma \in \R_+^{I\times J}} \sum_{i,j} c(x_i,x_j)\gamma_{i,j} - \epsilon H(\gamma) + \rho \text{KL}(\gamma \mathbf{1} \mid p) + \rho \text{KL}(\gamma^T \mathbf{1} \mid q)
\end{align}
\cite{feydyOptimalTransportDiffeomorphic2017a} then went on to compute $W_{\epsilon,\rho}$ with generalized Sinkhorn algorithm, and supply the gradient $\nabla W_{\epsilon,\rho}$ with respect to a discrete parameterization of $\varphi_{\#}\mu$ to minimize the energy $\sE$.

One possible extension is to extend lddmm+ot to the Bayesian setup. \cite{wassermannProbabilisticDiffeomorphicRegistration2014} uses variational method to compute an approximate posterior distribution $q(\varphi)$ that is close to the true posterior,
\begin{align}
    \text{KL}(q(\varphi) \Vert p(\varphi \mid X,Y) )
        = \text{KL}(q(\varphi) \Vert p(\varphi)) - \langle \log p(X,Y\mid \varphi) \rangle_{q} + \log p(X,Y)
\end{align}
where $\varphi$ is assumed to be a Gaussian process $p(\varphi) = \sG\sP( \mu_{\varphi}, k_{\varphi} )$. The paper made some derivations and proposed to minimize a lower bound of the form
\begin{align}
    \sE'(\varphi) 
        = \sR(\varphi) + \langle \sL(\varphi_{\#}\mu, \nu) \rangle_{q(\varphi)}
\end{align}
We can think of the second term as measuring the average data fidelity of a discrete random variable $\mu$ transformed by a random mapping $\varphi \sim q$ to another discrete random variable $\nu$. In case $\sL=W_{\epsilon,\rho}$, the question is if there is some simplification of the second term when we are taking expectation with respect to $q(\varphi)$ and/or if semi-discrete transport distance can be used instead somehow. I probably also need to parameterize velocity/transformation instead of optimizing for free variables of the pushforward $\varphi_{\#}\mu$ directly as \cite{feydyOptimalTransportDiffeomorphic2017a} did.



\newpage
\printbibliography 




\end{document}
