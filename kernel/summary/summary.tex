\documentclass[11pt]{article}
\input{\string~/.macros}
\usepackage[a4paper, total={6in, 9in}]{geometry}
\usepackage{bbm}
\usepackage{mathrsfs} % really cursive alphabets
\usepackage{graphicx}
\graphicspath{{./assets}}
\usepackage{hyperref}
\hypersetup{colorlinks=true, linktoc=all, linkcolor=blue, citecolor=red}
\usepackage[backend=bibtex,sorting=none]{biblatex}


% random variables
\newcommand\ry{\ensuremath{\mathsf{y}}}
\newcommand\rx{\ensuremath{\mathsf{x}}}
\newcommand\rb{\ensuremath{\mathsf{b}}}
\newcommand\rc{\ensuremath{\mathsf{c}}}
\newcommand\rz{\ensuremath{\mathsf{z}}}
\newcommand\ru{\ensuremath{\mathsf{u}}}
\newcommand\rbx{\ensuremath{\mathsf{\mathbf{x}}}}
\newcommand\rby{\ensuremath{\mathsf{\mathbf{y}}}}
\newcommand\rbu{\ensuremath{\mathsf{\mathbf{u}}}}

% boldsymbols
\renewcommand\bmu{\ensuremath{\boldsymbol{\mu}}}
\newcommand\bSigma{\ensuremath{\boldsymbol{\Sigma}}}

% optimization, classes of functions
\newcommand\scrF{\ensuremath{\mathscr{F}}}
\newcommand\scrS{\ensuremath{\mathscr{S}}}

% independence
\newcommand{\dperp}{\ensuremath{\perp\!\!\!\perp}}
\newcommand{\ndperp}{\ensuremath{\not\!\perp\!\!\!\perp}}

\begin{document}

\section{Support Vector Machines}

Support vector machine is a kernelized large margin linear classifier. For binary classification problem $\sY=\pc{-1,+1}$, we are interested in finding a linear decision bondary, parameterized by $w\in\R^d,b\in\R$, that separates the training data points by maximizing the worst case distance (margin) of each data point to the decision boundary. Given dataset $\pc{(x_i,y_i)}_{i=1}^n$, we are interested in solving the following quadratic programming problem,
\begin{align*}
    \min_{w,b}
        \;& \frac{1}{2} \norm{w}^2 \\
    \text{subject to}
        \;& y_i(w^Tx_i + b) \geq 1 \quad i=1,2,\cdots,n
\end{align*}
To derive the dual problem, we write the Lagrangian, 
\begin{align*}
    \sL(w,b,\alpha)
        &= \frac{1}{2} \norm{w}^2 + \sum_{i=1}^n \alpha_i \pb{ 1 - y_i(w^Tx_i+b) }
\end{align*}
where $\alpha = \pc{\alpha_i}_{i=1}^n$ are the dual variables. Solve for $\inf_{w,b} \sL(w,b,\alpha)$ to arrive at the dual objective. In particular, first order optimality condition gives $w = \sum_{i=1}^n \alpha_i y_i x_i$ and it must be that $0=\sum_{i=1}^n \alpha_i y_i$. Therefore, we arrive at the dual problem, 
\begin{align*}
    \max_{\alpha}
        \;& \sum_{i=1}^n \alpha_i - \frac{1}{2} \sum_{i=1}^n\sum_{j=1}^n \alpha_i \alpha_j y_i y_j x_i^Tx_j \\
    \text{subject to}
        \;& \alpha_i \geq 0 \quad i=1,2,\cdots, n
            \tag{dual feasibility} \\
        & \sum_{i=1}^n \alpha_i y_i = 0 
            \tag{from $\nabla_b \sL = 0$}
\end{align*}
The dual problem can be solved more efficiently than the primal problem using coordinate descent. The decision rule can be written entirely using dot products between input vectors,
\begin{align*}
    f(x)
        &= \sum_{i=1}^n \alpha_i y_i x_i^T x + b
\end{align*}
We observe that optimization as well as prediction uses input vectors via dot products only. We are motivated to use feature mapping $\phi$ to map input vectors to a higher dimensional space in hope that the lifted space is linearly separable. The kernel function $k: \R^d\times\R^d \to \R$ allows us to compute dot product $\phi(x_i)^T\phi(x_j)$ efficiently and even without ever defining the exact forms of $\phi$. We can substitute $k$ whenever inner product is used and arrive at a large margin classifier over implicitly defined nonlinear feature mapping $\phi$, i.e. support vector machines.




\newpage
\printbibliography 




\end{document}