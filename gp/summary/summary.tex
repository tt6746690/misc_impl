\documentclass[11pt]{article}
\input{\string~/.macros}
\usepackage[a4paper, total={6in, 9in}]{geometry}
\usepackage{bbm}
\usepackage{mathrsfs} % really cursive alphabets
\usepackage{graphicx}
\graphicspath{{./assets}}
\usepackage{hyperref}
\hypersetup{colorlinks=true, linktoc=all, linkcolor=blue, citecolor=red}
\usepackage[backend=bibtex,sorting=none]{biblatex}


% random variables
\newcommand\ry{\ensuremath{\mathsf{y}}}
\newcommand\rx{\ensuremath{\mathsf{x}}}
\newcommand\rb{\ensuremath{\mathsf{b}}}
\newcommand\rc{\ensuremath{\mathsf{c}}}
\newcommand\rz{\ensuremath{\mathsf{z}}}
\newcommand\ru{\ensuremath{\mathsf{u}}}
\newcommand\rbx{\ensuremath{\mathsf{\mathbf{x}}}}
\newcommand\rby{\ensuremath{\mathsf{\mathbf{y}}}}
\newcommand\rbu{\ensuremath{\mathsf{\mathbf{u}}}}

% boldsymbols
\renewcommand\bmu{\ensuremath{\boldsymbol{\mu}}}
\newcommand\bSigma{\ensuremath{\boldsymbol{\Sigma}}}

% optimization, classes of functions
\newcommand\scrF{\ensuremath{\mathscr{F}}}
\newcommand\scrS{\ensuremath{\mathscr{S}}}

% independence
\newcommand{\dperp}{\ensuremath{\perp\!\!\!\perp}}
\newcommand{\ndperp}{\ensuremath{\not\!\perp\!\!\!\perp}}
\addbibresource{GP.bib}

\begin{document}
 
\section{Gaussian Process}
 
\cite{carledwardrasmussenGaussianProcessMachine2006} introduces Gaussian Process regression as generalization of Bayesian regression. In linear regression setup, we assume output value $y\in\R$ is a linear function of inputs $x\in\R^d$, corrupted by iid normal noise.
\begin{align}
    y = f(x) + \epsilon
    \quad\text{where}\quad
    f(x) = w^Tx
    \quad\text{and}\quad
    \epsilon \sim \sN(0, \sigma_n^2)
\end{align}
The Bayesian setup considers $w\in\R^d$ as a random variable, endowed with prior $w\sim \sN(0,\Sigma_p)$. Using Bayes rule, we can find the posterior of weights given data, which is again a normal random variable $p(w \mid X, y) = \sN\left(w \; ; A^{-1}b, A^{-1} \right)$ where $A=\frac{1}{\sigma_n^2} X^TX + \Sigma_p^{-1}$ and $b = \frac{1}{\sigma_n^2}X^Ty$ and $X\in\R^{n\times d}, y\in\R^{n\times 1}$ are design matrices. For test point $x_*$, the predictive distribution of $f_*=f(x_*)$ is the average likelihood of $f_*$ under model $f(x;w)$ with respect to posterior of $w$.
\begin{align}
    p(f_*\mid x_*, y)
        = \int p(f_*\mid x_*,w) p(w\mid y) \, dw
\end{align}
We can think of the predictive distribution as a linear function $f_* = x_*^Tw$ of weights, a normal random variable, and therefore is normal. Therefore, $f_*\mid x_*,X,y \sim \sN( x_*^T A^{-1}b, x_*^T A^{-1}x_* )$. The natural extension to Bayesian linear regression is to kernelize it, for example assume a linear model in some feature space $f(x) = \phi(x)^Tw$. Instead of considering $w$ as a random variable, we can 

An alternative view point of considering $w$ as random variable is to consider the function class that $w$ parameterizes as a random variable.




 
 

\newpage
\printbibliography 




\end{document}